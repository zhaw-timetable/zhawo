% !TEX root = pa_doc.tex
\begin{markdown}

% TODO: rename file

# Architecture Technologies

*add image*

## Frontend
For the presentation of the web application to the user, we used React\cite{React}. We decided to use React because of its potential to write cleaner code with component based modularity. To minimise potential problems and to insure a a unidirectional data flow we chose to incorporate  the flux pattern into the architecture design, instead of using the typical Model View Controller (MVC) architecture.

The data for the front end is provided by the back end RESTful API service and retrievable through HTTP calls. Each fetch request is cached using a service worker. This eliminates the dependence on the network, ensuring an instant and reliable experience for our users. We built the service worker using Workbox\cite{Workbox}. Workbox is a library that implements in a set of service worker best practices and helps you get the most out of your service workers.

## Backend
The Sever-side of ZHAWo is implemented in NodeJS\cite{Node}. We use the ExpressJS Server to handle all the API calls.

% TODO: add image front-back api calls and stuff



\end{markdown}
