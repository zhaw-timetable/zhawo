\documentclass[english,12pt]{article}
% Für Deutsch
%\usepackage[utf8]{inputenc}
%\usepackage[ngerman]{babel}
\usepackage{blindtext} % Lorem ipsum
% Packages
\usepackage[a4paper,includeheadfoot,margin=2cm]{geometry} % Layout ,showframe --> to see border
\usepackage{caption}
\usepackage[hidelinks]{hyperref} %links within the document and for clickable URLs
\usepackage{graphicx}
\usepackage[export]{adjustbox} % to beable to position images
\usepackage{amsmath} % equations
\usepackage{listings} % codeblocks
\usepackage{dirtytalk} % use quotations

\usepackage[backend=bibtex,style=ieee]{biblatex} % or verbose-trad2
\bibliography{source.bib} % add sources to source.bib file

\usepackage{titling} % So you can use theauthor

\usepackage[hybrid]{markdown} % Used to convert markdown

% Image
\graphicspath{{img/}}
\usepackage{float} % Allows putting an [H] (place HERE & not where LaTeX wants to put)

% Header & Footer
\usepackage{fancyhdr}

\usepackage{siunitx} % Required for alignment
\sisetup{
	round-mode          = places, % Rounds numbers
	round-precision     = 2, % to 2 places
}
\usepackage{booktabs} % For prettier tables

% TODO: better Title page
% Document information
\title{ZHAWo - Platform Independent Timetable App}
\author{Bachmann Dominik, Visser Julian}

% Header & Footer
\pagestyle{fancy}
\fancyhf{}
\lhead{PA - HS 2018}
\chead{}
\rhead{ZHAWo}
\lfoot{}
\cfoot{\thepage}
\rfoot{}

\begin{document}
	\setlength{\parindent}{0in} % removes default indent of new paragraph


	% !TEX root = pa_doc.tex

%----------------------------------------------------------------------------------------
%	TITLE PAGE
%----------------------------------------------------------------------------------------
\begin{titlepage} % Suppresses displaying the page number on the title page and the subsequent page counts as page 1
	\newcommand{\HRule}{\rule{\linewidth}{0.5mm}} % Defines a new command for horizontal lines, change thickness here

	\center % Centre everything on the page

	%------------------------------------------------
	%	Headings
	%------------------------------------------------

	\includegraphics[width=0.3\textwidth, left]{./assets/zhawLogo.jpeg}\\[1cm]

	\textsc{\Large Prijekt Arbeit}\\[0.5cm] % Major heading such as course name

	\textsc{\large HS 2018}\\[0.5cm] % Minor heading such as course title

	%------------------------------------------------
	%	Title
	%------------------------------------------------

	\HRule\\[0.5cm]

	%------------------------------------------------
	%	Logo
	%------------------------------------------------

	\includegraphics[width=0.4\textwidth]{./assets/zhawoLogo.png}\\[0.5cm]

	\textsc{\large \textbf{ZHAWo} \\[0.2cm]
									--- \\[0.3cm]}
	% Title of your document
	\textsc{\large Platform Independent Timetable App}\\[0.5cm]


	\HRule\\[1cm]

	%------------------------------------------------
	%	Author(s)
	%------------------------------------------------

	\begin{minipage}{0.4\textwidth}
		\begin{flushleft}
			\large
			\textit{Authors}\\
			Bachmann Dominik \\
			Visser Julian % Your name
		\end{flushleft}
	\end{minipage}
	~
	\begin{minipage}{0.4\textwidth}
		\begin{flushright}
			\large
			\textit{Supervisor}\\
			Meier Andreas % Supervisor's name
		\end{flushright}
	\end{minipage}

	%----------------------------------------------------------------------------------------

	%------------------------------------------------
	%	Date
	%------------------------------------------------
	\vfill
	{\large\today} % Date, change the \today to a set date if you want to be precise

\end{titlepage}


	\pagenumbering{gobble}


	%Index
	\newpage
	\pagenumbering{roman}
	\tableofcontents
	\newpage
	%Abstract
	% !TEX root = pa_doc.tex
\begin{abstract}
\begin{markdown}

\noindent Where is my next lecture? Is there a free room for me to study in? What is the Lunch Menu today?

\noindent These are just a few of the many questions Students have to deal with daily and getting an anwser to these questions isn't always that easy.

\noindent Our vision for zhawo is to build a modern application that helps the students of the ZHAW with their every day study needs. Using agile development, the goal is to work closely with students to provide an application that is tailored to their specific needs.

\noindent We want to provide our Users a fast, reliable and engaging experience across all Platforms. That is why we have chosen to develop zhawo as a Progressive Web App. Progressive Web Apps are web apps that behave like native apps. That means the application can be accessed by any device that has a browser, whilst still providing the user with the look and feel of a native app.
\end{markdown}
\end{abstract}


	% Main body of your document
	\newpage
	\pagenumbering{arabic}

	%\markdownInput{example.md}

	% !TEX root = pa_doc.tex
\begin{markdown}

# Introduction

Students of the Zürcher Hochschule für Angewandte Wissenschaften (ZHAW) have different applications and sites to get
information about their schedules and menu plans of their mensa. 
%TODO change this sentence below -> less complicated

However, apart from the official Android application \cite{DUMMY},
they are either not well suited for use on mobile devices in the case of the official schedule site \cite{DUMMY} or feel
outdated.
%TODO write something about issue of maintaining code for both Android and Iphone

By using progressive web application (PWA) technologies \cite{DUMMY} --> a platform independent application...
%TODO write more

## Goals + Primary Functions
%TODO how to make &

--> List from Readme

\end{markdown}

  \newpage

	% !TEX root = pa_doc.tex
\begin{markdown}

% TODO: rename file

# Architecture Technologies

*add image*

## Frontend
For the presentation of the web application to the user, we used React\cite{React}. We decided to use React because of its potential to write cleaner code with component based modularity. To minimise potential problems and to insure a a unidirectional data flow we chose to incorporate  the flux pattern into the architecture design, instead of using the typical Model View Controller (MVC) architecture.

The data for the front end is provided by the back end RESTful API service and retrievable through HTTP calls. Each fetch request is cached using a service worker. This eliminates the dependence on the network, ensuring an instant and reliable experience for our users. We built the service worker using Workbox\cite{Workbox}. Workbox is a library that implements in a set of service worker best practices and helps you get the most out of your service workers.

## Backend
The Sever-side of ZHAWo is implemented in NodeJS\cite{Node}. We use the ExpressJS Server to handle all the API calls.

% TODO: add image front-back api calls and stuff



\end{markdown}

  \newpage

  % !TEX root = pa_doc.tex
\begin{markdown}
# Technologies



## PWA

## React
React\cite{React} was originaly developed by Facebook and is one of the most popular UI librarys. It allows you to create reusable Compents using JSX, a syntax extension to JavaScript.
The idea behind React is to design simple views for each state of the application. Doing so allows it to only update and render components that need to be changed, thereby improving performance immensely.


### Flux

Instead of using the typical Model View Controller (MVC) architecture, we chose to incorporate Flux\cite{OurReadme} into the architecture design. Flux is a application pattern developed by Facebook. It's goal is to insure a a unidirectional data flow in React apps. The use of this practice enhances the quality and performance of the code by improving the data consistency. It is the optimal architecture for the use of React. Although, completely new and never used by any of the team members, the decision was made to implement the Flux pattern to achieve the best possible results.

\begin{figure}[H]
  \includegraphics[width=10cm, center]{./assets/flux.png}
  \caption{Flux Model{\cite{FluxModel}}}
\end{figure}



% TODO: bild ref https://facebook.github.io/flux/img/flux-simple-f8-diagram-with-client-action-1300w.png

In Flux, the dispatcher is a singleton that directs the flow of data to ensure that updates do not cascade, which would lead to unpredictable behaviour. When a user interacts with a React view, the view sends an action through the dispatcher, which notifies the stores that hold the application’s data. When the stores change state, the view gets notified and changes accordingly.

## NodeJS
Node.js is a JavaScript runtime environment, designed to build scalable network applications\cite{Node}.
*öpis vo da https://nodejs.org/en/about/* blocking and so on
### Express
Express\cite{Express} is a minimal and flexible web application framework for Node.js. It provides HTTP utility methods and allows you to create a robust API.
With just a few lines of Code you have a WebServer up and running.

% TODO: failt
\begin{lstlisting}
const express = require('express');
const app = express();

app.get('/', (req, res) => {
    res.send('An alligator approaches!');
});

app.listen(3000, () => console.log('Gator app listening on port 3000!'));
\end{lstlisting}

\end{markdown}

  \newpage

  % !TEX root = pa_doc.tex
\begin{markdown}

# Development

Explain how agile development was used, Github Projects, Travis, Codecov, Heroku

## Agile

## Continuous Integration \& Deployment
A multi-branch system on Jenkins [14] is used to continuously build every open branch and pull request with each commit. This approach ensures that failing tests, builds or merges are detected as early as possible. After a successful build on Jenkins the front end gets deployed on an apache server and the back end on an apache tomcat server on the local cloudlab virtual machine from ZHAW. The web application is accessible on http://client101.cloudlab.zhaw.ch/ via the ZHAW network (and VPN). The continuous integration and deployment run fully automatically via Jenkins.
To meet the definition of done the quality of the develop-branch is summarized on a SonarQube [15] server running on the cloudlab virtual machine from ZHAW. With this approach the quality of the application was assessed continuously according to preset quality gates.

\end{markdown}

  \newpage

  % !TEX root = pa_doc.tex
\begin{markdown}

# Discussion
Over the past Semster we built fast and reliable prototype of ZHAWo as a PWA. Our inital goals were met. Nevertheless, our application still needs further optimization before it can be used in a productive environment. There are also additional features that we would like to implement to further improve the experience of our users.

vielicht no was vowege hend scho idea vo de ersi tester becho ...

Outlook -> from prototype to final app? possible features, etc, changes to approach..

\end{markdown}

  \newpage

	\section{Appendix}

	\listoffigures
	\listoftables
	%\renewcommand\refname{Literaturverzeichnis}% rename --- für deutsh
	\printbibliography  % Uses source.bib to make ref table

\end{document}
