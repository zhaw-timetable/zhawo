% !TEX root = ba_doc.tex
\begin{markdown}
\section{Results} \label{results}

## Application

The user interface was designed to provide students with the familiar look and feel of a native mobile application. We chose a minimalistic design approach. The focus was on displaying relevant information without any distracting noise or clutter and a fast and intuitive user experience. Users can choose between a light theme and a dark theme (Figures \ref{fig:LoginFigure}, \ref{fig:MenuFigure} and \ref{fig:EventsFigure}).
On desktop devices the drawer is always open and contains the navigation. (Figures \ref{fig:DesktopSchedule} and \ref{fig:DesktopRoomsearch})

A detailed interaction flow with ZHAWo is described in the following sections.
\begin{figure}[H]
  \includegraphics[width=13.5cm, center]{./figures/desktop_schedule.png}
  \captionsetup{width=15.5cm}
  \caption[Desktop Schedule]{Desktop view of a student schedule.}
  \label{fig:DesktopSchedule}
\end{figure}

\vspace{-3ex} % used to reduce padding after figure

\begin{figure}[H]
  \includegraphics[width=13.5cm, center]{./figures/desktop_roomsearch.png}
  \captionsetup{width=15.5cm}
  \caption[Desktop Roomsearch]{Desktop view of the free rooms on the 4. Floor in TE.}
  \label{fig:DesktopRoomsearch}
\end{figure}

### General

On a users first visit to ZHAWo, a prompt is shown to enter their ZHAW username (Figure \ref{fig:LoginFigure} A). A list of all available student and professor names is loaded and on user input, suggestions from that list are displayed. Since there is no persistent user specific data, setting up an account is not needed. The ZHAW username is required to fetch the correct timetable. This username is stored and on the next visit, the initial screen is skipped and users have immediate access to their timetable.

\begin{figure}[H]
  \includegraphics[width=16cm, center]{./figures/login_figure.png}
  \captionsetup{width=15.5cm}
  \caption[General user interface]{\textbf{General user interface}: \textbf{A}: On first visit, users are prompted to enter (1) and select (2) their ZHAW username. \textbf{B}: Through a collapsible application drawer, users have access to application settings (3) to change the application theme and timetable display mode. Users can clear their stored ZHAW username through a logout button (4). \textbf{C}: Users can switch between timetable, mensa menu, room search and student events contexts through labeled icons on a bottom navigation bar (5).}
  \label{fig:LoginFigure}
\end{figure}

Through a collapsible application drawer, users have access to some basic settings like switching between display modes of the timetable and changing the application theme (Figure \ref{fig:LoginFigure} B). The two timetable display modes show the user's timetable for either a single day or a full week. The saved ZHAW username can be cleared by clicking the logout button at the bottom of the application drawer.

Switching between contexts (corresponding to the four primary functions timetable, mensa menus, room search and student events) is done through labeled icons on a navigation bar at the bottom of the screen (Figure \ref{fig:LoginFigure} C). The application state such as selected date or time frame for the room search are preserved between context switches.

\newpage

### Timetable

By default, the timetable is displayed in a day view (Figure \ref{fig:TimetableFigure1} A). Individual events are displayed in their corresponding time slots. The time slots are displayed on the left with a subtle indicator for the current time slot. If there are events that run overlap in time, they are displayed next to each other in the day view. To navigate between different dates, users can swipe to the right to advance one day or swipe to the left to go to the previous day respectively. Above the timetable, a navigation bar is displayed which highlights the currently displayed day or week. As an alternative to the swipe interactions, users can directly select a date in the navigation bar or jump one week back or forward by clicking on the arrow buttons. To quickly jump back to the timetable of day current day or week respectively, users can click on the calendar icon in a context action area on the top right of the screen (Figure \ref{fig:TimetableFigure1} B).

In addition to the day view, users can choose to display their timetable for a whole week. Since display space is limited especially on small mobile devices, overlapping events are shown as an indicator below the longest running event during that time frame as seen in \ref{fig:TimetableFigure1} B.

When clicking on a specific event, a modal pops up displaying detailed information about that event (Figure \ref{fig:TimetableFigure1} C).

\bigskip

\begin{figure}[H]
  \includegraphics[width=16cm, center]{./figures/timetable_figure1.png}
  \captionsetup{width=15.5cm}
  \caption[Timetable user interface]{\textbf{Timetable user interface}: \textbf{A}: Day view of a users timetable with overlapping events from 10:00 to 11:35. Navigation bar to navigate to specific dates and to navigate between weeks (1). Alternatively, navigation between days or weeks can be achieved through swipe gestures. \textbf{B}: Week view of a users timetable with overlapping event indicators and context action area button to jump to current date (2). \textbf{C}: Detail modal of a specific event. Is opened when user clicks on event from either day or week view.}
  \label{fig:TimetableFigure1}
\end{figure}

\newpage

Users can also search for and display other timetables than their own. Clicking on the search icon button in the context action area opens a modal. Users can search for and display timetables of other students, lecturers, classes, courses or rooms (Figure \ref{fig:TimetableFigure2} A and B). The currently displayed search timetable can be cleared by clicking the indicator in the context action area (Figure \ref{fig:TimetableFigure2} C).

\bigskip

\begin{figure}[H]
  \includegraphics[width=16cm, center]{./figures/timetable_figure2.png}
  \captionsetup{width=15.5cm}
  \caption[Timetable search user interface]{\textbf{Timetable search user interface}: \textbf{A}: Default user timetable with search icon button in context action area to open search modal (1). \textbf{B}: Search modal with suggestions for students, lecturers, classes, courses and rooms (2). \textbf{C}: Search timetable for specific class with indicator and clear button (3) in context action area.}
  \label{fig:TimetableFigure2}
\end{figure}

\newpage

### Mensa menus

The mensa menus are displayed as a list of available menus with their category, menu name and description and ZHAW internal prices. The title indicates the mensa where the currently displayed menu is available. By clicking the mensa icon in the context action area, users can select all mensas with menu information available (Figure \ref{fig:MenuFigure} A and B). Navigation between menus of different days is consistent with the navigation for the timetable context, with a navigation bar in addition to swipe gestures.

\bigskip

\begin{figure}[H]
  \includegraphics[width=16cm, center]{./figures/menu_figure.png}
  \captionsetup{width=15.5cm}
  \caption[Mensa menu user interface]{\textbf{Mensa menu user interface}: \textbf{A}: Mensa menu for the Technikum mensa with menu category, name, description and ZHAW internal prices. Title indicates currently displayed mensa which can be changed through the mensa icon in the context action area (1). \textbf{B}: Mensa selection (2). \textbf{C}: Dark theme mensa menu display.}
  \label{fig:MenuFigure}
\end{figure}

\newpage

### Room search

To search for currently unoccupied rooms at the ZHAW School of Engineering, users can enter a time range during which they want to use the room (Figure \ref{fig:RoomSearchFigure} A). Once they've entered the start and end time, they can press the search button and the rooms will be filtered to only show rooms that are unoccupied during the choosen time slots. To browse the free rooms, the users can navigate through a map of the School Of Engineering campus showing all the buildings. Buildings highlighted in blue have at least one free room during the entered time range (Figure \ref{fig:RoomSearchFigure} B). To see which rooms are free, users can click on a building on the map, and a floor plan of that building will appear with labeled and highlighted free rooms. They can navigate through the different floors with the buttons above the floor plans. The individual floor plans show which rooms are free and where they are (Figure \ref{fig:RoomSearchFigure} C). To switch to a different building, users can navigate back to the campus overview map through the School Of Engineering button next to the floor navigation.

\bigskip

\begin{figure}[H]
  \includegraphics[width=16cm, center]{./figures/roomsearch_figure1.png}
  \captionsetup{width=15.5cm}
  \caption[Room search user interface]{\textbf{Room search user interface}: \textbf{A}: Initial screen of the room search without an active search with input for start and end time and search button (1). \textbf{B}: Campus overview map of the School Of Engineering with buildings highlighted in blue that contain a free room in the entered time range. By clicking on a building on the map, users get directed to the respective floor plans of that building (2). \textbf{C}: Detailed floor plan of TE building with free rooms highlighted and labeled. Navigation through different floors with buttons above the map (3).}
  \label{fig:RoomSearchFigure}
\end{figure}

\newpage

### Student events

Users can browse vszhaw news as a chronological list of blog posts. The full news posts are linked and can be reached by clicking on the individual posts. If there is an upcoming event in the official vszhaw calendar, it will be featured above the news feed (Figure \ref{fig:EventsFigure} A and B). Additionally, student events are displayed as a banner in the timetable context of their date (Figure \ref{fig:EventsFigure} C). Both the banner and student event feature are linked to more a detailed view of the event on the vszhaw website.

\bigskip

\begin{figure}[H]
  \includegraphics[width=16cm, center]{./figures/events_figure.png}
  \captionsetup{width=15.5cm}
  \caption[Student events user interface]{\textbf{Student events user interface}: \textbf{A}: Vszhaw student events (1) and news feed (2). \textbf{B}: Dark theme version of vszhaw news feed. \textbf{C}: Student event banner in the timetable context (3).}
  \label{fig:EventsFigure}
\end{figure}

\newpage

## Test coverage

To test our application we used the framework Jest \cite{Jest} for both front- and backend. To test the React components, we used the GUI testing framework \cite{Enzyme} in addition to Jest.

For the express server application, we achieved a test coverage of 58\%. For the frontend React application 74\% of the code was covered, resulting in an overall project test coverage of about 71\% \cite{OurCoverage}. 
//TODO: rewrite this after coverage is done. The coverage is lower than we had planned and is something we aim to improve when going from application prototype to production ready application. This can be in part attributed to the prototypical implementation of the primary functions menu plan, room search and student events. Another aspect is that the current JavaScript framework environment is very dynamic and fast paced, which on one hand offers many advantages, on the other hand, best testing practices have not been established. For example, it was challenging to isolate and test the different components of the Flux architecture in the frontend. Therefore, we decided to put a lot of focus on establishing better testing practices in the next stage of this project.

## User feedback

To distribute our application and receive user feedback, we demonstrated our application to random students on the School Of Engineering campus. We asked the students to use our application for a bit and then fill out a survey. The survey contained questions about the usefulness, performance and design of ZHAWo as well as open questions about general feedback and improvement ideas. Out of 37 total survey participants, 75.7\% (28) rated the usefulness of ZHAWo with 5 out of 5 points (Figure \ref{fig:BarUsefulness}) and the average rating of the usefulness with 4.75 out of 5 was very high. The performance was rated with an average of 4.62 out of 5 points (Figure \ref{fig:BarPerformance}). The design of the application was rated with a lower point score of on average 4.24 out of 5 points (Figure \ref{fig:BarDesign}). The design rating is rather subjective, with some participants pointing out that they much prefer the simplistic design of ZHAWo over the official applications, while others list the design as an area where ZHAWo could improve a lot. Some of the lower ratings for the design can be attributed to the fact that especially on Huawei phones, the floor plans were not always displayed correctly.

In general, the room search feature and it's implementation with floor plans that display the locations of free rooms was received very positively, with most participants pointing out that feature as very useful and something they had been wishing for. A few participants also asked if the room search could be expanded to allow them to search for free rooms on a specific date. Others said that it should be expanded to also include other locations and not just the School Of Engineering campus at the Technikum.

The way we handled the display of overlapping events for timetables was also well received and was noted to be more readable compared to the official CampusInfo app.

In addition to getting feedback on our application, we also wanted to get a sense of how familiar people are with the concept of Progressive Web Applications. A majority of the survey participants (78.4\%) have not previously heard about PWAs. This coincided with what expected. PWAs are a new technology and people are only just starting to learn about them. 

\begin{figure}[H]
  \includegraphics[width=13cm, center]{./figures/bar_1.png}
  \captionsetup{width=15.5cm}
  \caption [Usefulness Bar Diagram]{Distribution of the responses regarding the usefulness of ZHAWo.}
  \label{fig:BarUsefulness}
\end{figure}

\vspace{-5ex} % used to reduce padding after figure

\begin{figure}[H]
  \includegraphics[width=13cm, center]{./figures/bar_2.png}
  \captionsetup{width=15.5cm}
  \caption [Performance Bar Diagram]{Distribution of the responses regarding the performance of ZHAWo.}
  \label{fig:BarPerformance}
\end{figure}

\vspace{-5ex} % used to reduce padding after figure

\begin{figure}[H]
  \includegraphics[width=13cm, center]{./figures/bar_3.png}
  \captionsetup{width=15.5cm}
  \caption [Design Bar Diagram]{Distribution of the responses regarding the design of ZHAWo.}
  \label{fig:BarDesign}
\end{figure}

## Metrics

Google provides the automated tool Lighthouse \cite{Lighthouse} to audit and rate PWAs. Besides providing performance metrics, Lighthouse can also be used to test various PWA aspects, including if the application still responds without network connection. ZHAWo achieved a performance rating of 100 out of 100, confirming the user feedback that the app has a very good performance. ZHAWo also passed all automated PWA checks confirming that it performs well even on slow network and caching of offline content through the service worker works as intended. Manual tests confirm that as long as a resource such as a timetable or a mensa menu were previously fetched, the application still functions even without any connection to the internet. A detailed report of the Lighthouse audit can be found on the Github Repository \cite{OurGithub}.

During the three weeks we published the application to a broader range of students, outside our small testing and feedback group. To date ZHAWo has had about 70 unique users. We were also positively surprised to see that the users were using the application for everything. To date the schedule has been used 411 times, the room search 295 times, the mensa 373 times and vszhaw 186 times. This user base consists of the participants of the survey and people who either heard about the application through those participants. It also consists of our testing group users. We expect the number of individual users to increase at the start of the next semester when we will send out an email about ZHAWo in collaboration with the vszhaw.

\bigskip

\begin{figure}[H]
  \includegraphics[width=13cm, center]{../../metrics/ZHAWoLighthousereportBAPreformance.png}
  \captionsetup{width=15.5cm}
  \caption [Lighthouse Preformance]{Preformance Results from Lighthouse report.}
  \label{fig:LighthousePreformance}
\end{figure}

\newpage

\end{markdown}

