% !TEX root = pa_doc.tex
\begin{markdown}
# Technologies



## PWA

## React
React\cite{React} was originaly developed by Facebook and is one of the most popular UI librarys. It allows you to create reusable Compents using JSX, a syntax extension to JavaScript.
The idea behind React is to design simple views for each state of the application. Doing so allows it to only update and render components that need to be changed, thereby improving performance immensely.


### Flux

Instead of using the typical Model View Controller (MVC) architecture, we chose to incorporate Flux\cite{OurReadme} into the architecture design. Flux is a application pattern developed by Facebook. It's goal is to insure a a unidirectional data flow in React apps. The use of this practice enhances the quality and performance of the code by improving the data consistency. It is the optimal architecture for the use of React. Although, completely new and never used by any of the team members, the decision was made to implement the Flux pattern to achieve the best possible results.

\begin{figure}[H]
  \includegraphics[width=10cm, center]{./assets/flux.png}
  \caption{Flux Model{\cite{FluxModel}}}
\end{figure}



% TODO: bild ref https://facebook.github.io/flux/img/flux-simple-f8-diagram-with-client-action-1300w.png

In Flux, the dispatcher is a singleton that directs the flow of data to ensure that updates do not cascade, which would lead to unpredictable behaviour. When a user interacts with a React view, the view sends an action through the dispatcher, which notifies the stores that hold the application’s data. When the stores change state, the view gets notified and changes accordingly.

## NodeJS
Node.js is a JavaScript runtime environment, designed to build scalable network applications\cite{Node}.
*öpis vo da https://nodejs.org/en/about/* blocking and so on
### Express
Express\cite{Express} is a minimal and flexible web application framework for Node.js. It provides HTTP utility methods and allows you to create a robust API.
With just a few lines of Code you have a WebServer up and running.

% TODO: failt
\begin{lstlisting}
const express = require('express');
const app = express();

app.get('/', (req, res) => {
    res.send('An alligator approaches!');
});

app.listen(3000, () => console.log('Gator app listening on port 3000!'));
\end{lstlisting}

\end{markdown}
